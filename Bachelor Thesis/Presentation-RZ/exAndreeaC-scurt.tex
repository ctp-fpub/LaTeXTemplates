
%%%%%%%%%%%%%%%%%%%%%%%%%%%%%%%%%%%%%%%%%%%%%%%%%%%%%%%%%%%%%%%%%%%%%%%%%%%



\documentclass[a4,compress]{beamer}
\usepackage{beamerthemesplit}
\usepackage{ae}
\usepackage{graphicx}
\usepackage{psfrag}
\usepackage{epsfig}
\usepackage{rotating}
\usepackage{amssymb,amsmath}
\usepackage{german}
\usepackage[T1]{fontenc}
\usepackage[latin1]{inputenc}

\newcommand{\cl}{\mathcal{l}}

\usepackage{color}


\definecolor{darkblue}{rgb}{0.2,0.2,0.6}
\definecolor{darkdarkblue}{rgb}{0.12,.03,0.51}

\definecolor{pale}{rgb}{1,0.99,0.85}


\definecolor{lightblue}{rgb}{0.75,0.82,0.9}


\mode<presentation>
{
  \usetheme{Warsaw}%{Singapore}%{Warsaw}%{Malmoe}%{Copenhagen}%{Boadilla}
  %\usefonttheme[onlylarge]{structuresmallcapsserif}
  %\setbeamercovered{transparent}
  \setbeamercovered{white}
  %\setbeamercolor{frametitle}{fg=black,bg=white}
  
}


%\useoutertheme{infolines}
%\usecolortheme{rose}
%\useoutertheme{split}
%\useoutertheme{smoothbars}
%{lily}{dolphin}{seahorse}
%\usecolortheme{lily}
%\useoutertheme{sidebar}

\useinnertheme{rounded}
%\usecolortheme{rose}


%\usecolortheme{seahorse}


% -------------------------------
% % Titel mit Farbe, starke Farben
%\usecolortheme{whale}
% -------------------------------
% % Titel mit Farbe, schwache Farben
% \usecolortheme{seahorse}
% -------------------------------
% % Titel ohne Farbe, starke Farben
% \usecolortheme{dolphin}

%%%%%%%%%%%%%%%%%%%%slide 1  %%%%%%%%%%%%%%%%%%%%%%%%%%%%%%%%%%%%%%


\title[Titlu scurt lucrare]{Titlu complet lucrare}
       
       
\author{Andreaa}
\institute{Universitatea din Bucuresti}
\date{\small  Bucuresti, 30 iunie 2010}

\begin{document}




%%%%%%%%%%%%%%%%%%%%%%%%%%%%%% Slide 1 %%%%%%%%%%%%%%%%%%%%%%%%%%%%%%%%%%%%%%%
% Aici apare titlul 


%fiecare slide incepe cu \begin{frame} si se termina cu \end{frame}


\begin{frame}
\frametitle{}
 \setbeamercolor{title}{fg=darkdarkblue,bg=lightblue}
\setbeamercolor{date}{fg=darkdarkblue,bg=lightblue}
 \titlepage
\end{frame}


%%%%%%%%%%%%%%%%%%%%slide 2  %%%%%%%%%%%%%%%%%%%%%%%%%%%%%%%%%%%%%%


%Aici apare rezumatul sectiunilor pe care le ai

\begin{frame}[plain]
  \frametitle{Outline}
  \small
  \tableofcontents[hideallsubsections]
\end{frame}




%%%%%%%%%%%%%%%%%%%%slide  3 %%%%%%%%%%%%%%%%%%%%%%%%%%%%%%%%%%%%%%

%%%%%%%%%%%%%%%%%%%%%%%%%%%%%%%%%%%%%%%%%%%%%%%%%%%%%%%%%%%%%%%%%%%%%%%%%%%
%                                                                         %
%                       section:  Introduction                          %
%                                                                         %
%%%%%%%%%%%%%%%%%%%%%%%%%%%%%%%%%%%%%%%%%%%%%%%%%%%%%%%%%%%%%%%%%%%%%%%%%%%


\section[Introduction]{Introduction  and motivation}
%\tableofcontents[currentsection]
\subsection{Motivation}
\begin{frame}
  \frametitle{Motivation}
  
 
\begin{itemize}
\item Recent literature reveals:
\begin{itemize}
	\item an increased interest in postulating electrodynamics with {\color{violet} Jefimenko's equations};
	\item lengthy and complicated calculations for the {\color{blue} multipolar expansions} of the electromagnetic field;
	\item ambiguities when discussing the {\color{red} radiative systems}.
\end{itemize}
\pause %iti lasa pauza in prezentare

\item Up to what extend should one use {\color{violet} Jefimenko's equations} for describing electrodynamics, in general, and the {\color{red} radiated field}, in particular?
\end{itemize}

\end{frame}



%%%%%%%%%%%%%%%%%%%%% slide 4 %%%%%%%%%%%%%%%%%%%%%%%%%%%%%%%%%%%%%%
\subsection{Introduction}

\begin{frame}
  \frametitle{{\it Traditional} Electrodynamics}
\begin{itemize}
\item Aici un alt subsection pt. Introduction
\end{itemize}

\end{frame}

%%%%%%%%%%%%%%%%%%%%% slide 5 %%%%%%%%%%%%%%%%%%%%%%%%%%%%%%%%%%%%%%

%%%%%%%%%%%%%%%%%%%%%%%%%%%%%%%%%%%%%%%%%%%%%%%%%%%%%%%%%%%%%%%%%%%%%%%%%%%
%                                                                         %
%                       section:  Multipolar expansion of the electromagnetic field                          %
%                                                                         %
%%%%%%%%%%%%%%%%%%%%%%%%%%%%%%%%%%%%%%%%%%%%%%%%%%%%%%%%%%%%%%%%%%%%%%%%%%%

\section[Titlu scurt]{Titlu lung al sectiunii}
%\tableofcontents[currentsection]
\subsection{Electric and magnetic moment}
\begin{frame}
 Alta sectiune.... Fara titlu la slide. \\
\pause
Ecuatie (in prezentari nu sunt neaparat numerotate):
\begin{equation*} % * pentru ecuatie nenumerotata
\partial \psi = ....
\end{equation*}

\end{frame}

%%%%%%%%%%%%%%%%%%%%slide  7 %%%%%%%%%%%%%%%%%%%%%%%%%%%%%%%%%%%%%%

%%%%%%%%%%%%%%%%%%%%slide  11 %%%%%%%%%%%%%%%%%%%%%%%%%%%%%%%%%%%%%%

\section[Conclusion]{Discussion and conclusions}
%\tableofcontents[currentsection]
%\subsection{Electric and magnetic moment}
\begin{frame}

\begin{itemize}
	\item Concluzie 1
	\item Concluzie 2, etc.......
\end{itemize}

\end{frame}



\end{document}
