\documentclass[../thesis.tex]{subfiles}

%!TeX spellcheck = en-GB

\begin{document}

\addcontentsline{toc}{chapter}{Introduction}
\chapter*{Introduction}

The behaviour of non-linear systems, at classical and quantum level, rises
fundamental questions and is important in several applications.
In classical mechanics, once the Hamiltonian systems become non-integrable,
a sensitive dependence of the dynamics on the initial conditions manifests.
This corresponds to chaotic dynamics and leads to several specific phenomena.
In particular, the phase space is no longer filled by tori and depending on the
energy, several trajectories will have an erratic behaviour in the available volume.
An important question which is intensively debated in literature is related to
the way in which the corresponding quantum system reflects this classical properties.
The purpose of the present work is to explore the features of the energy levels
statistics in relation with the departure from the integrable case and also as
a function of excitation energy. We try to obtain a correlation between the
global phase space structure in a given energy domain and the associated
energy level distribution for the corresponding quantum system.

In~\cref{chap:classical-chaos} we review the basic concepts from classical
Hamiltonian dynamics which are useful to the characterisation of chaotic dynamics.

In~\cref{chap:quantum-chaos}, after a brief review of the principles of quantum mechanics,
we discuss the nearest neighbour distributions and their relation to the
non-integrability.

\Cref{chap:procedure} is devoted to a detailed description of the numerical procedures
and to the analysis method of the level distributions performed with the aid of a
\texttt{Python} program.

\Cref{chap:results} presents the main results of our investigations.
We discuss the possible deviations from the Wigner distribution by proposing an
effective level distribution which can interpolate between the Wigner and Poissonian one.
We show that the behaviour of this distribution can reflect the energy variations
in phase space. Moreover, it has peculiar behaviour at the variations of the parameter
with the departure from integrability.

In the last chapter, the main conclusions of our work are summarised.

\section*{Acknowledgments}
I would like to thank my advisers, Prof.~dr.~Virgil Băran and Lect.~dr.~Roxana Zus,
for their help and for delivering one of the most interesting lectures.

\end{document}
