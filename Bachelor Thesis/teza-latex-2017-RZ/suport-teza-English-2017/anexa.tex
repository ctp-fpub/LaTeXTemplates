

%anexa este optionala, dar pot fi si mai multe anexe, fiercare cu titlul ei, la chapter

\begin{appendices}
	%%%%%%%%%%%%%%%%%%%%
	\chapter{Integrale} 
	\label{anexa1}
	%%%%%%%%%%%%%%%%%%
	
	
	\begin{table}[h]
		\centerline{
			\begin{tabular}{|c|c|c|c|} %cate coloane are
				\hline
				% & &  & \\
				Rule & No. of points/ $[a,b]$ & $h$ & Formula for $I=\displaystyle\int_a^b{f(x)} dx$ \\
				% & Interval $[a,b]$& & $I=\displaystyle\int_a^b{f(x)}$ \\
				\hline
				\hline
				& &  & \\
				Simpson &  $3$ points: $[x_0,x_2]$& $h=\displaystyle\frac{b-a}{2}$ & $I \simeq \displaystyle\frac{h}{3}(f_0 + 4f_1 + f_2) + {\cal O}(h^5) $ \\ 
				\hline				
				& &  & \\ %rand gol
			coloana 1 &  2 & 3 & 4 \\ 
				\hline
			\end{tabular}
		}
		\caption{\label{cfni} Closed formulas for numerical integration.}
	\end{table}
	
	%%%%%%%%%%%%%%%%%%
	\chapter{Alta anexa}
	\label{anexa2}
	%%%%%%%%%%%%%%%%
	
	
	
	
\end{appendices}